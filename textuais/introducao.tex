\introducao{
    \section{Contextualização}
      
     
    \section{Problema}
     

    \section{Objetivos}
        \subsection{Objetivo Geral}
            \par Este trabalho tem como objetivo, realizar um estudo na literatura em relação as principais arquiteturas de software e demonstrar suas semelhanças e diferenças em relação a arquitetura limpa. Definindo motivações que levam o uso da mesma em um projeto de software.
            
        \subsection{Objetivos Específicos}
            \par Para a realização deste trabalho de conclusão de curso serão buscados os seguintes objetivos específicos:

            \begin{itemize}

                \item Realizar uma análise na literatura sobre as principais arquiteturas de software.
                \item Realizar uma análise na literatura sobre padrões de design relevantes para a implementação da arquitetura limpa.
                \item Realizar uma análise na literatura sobre a arquitetura limpa.
                \item Demonstrar os principais pontos que levam a utilização da arquitetura limpa em relação a outras arquiteturas de software.
            \end{itemize}
            
    \section{Metodologia}
        \par A metodologia utilizada neste trabalho de conclusão de curso tem natureza básica pura, com o objetivo de realizar um estudo descritivo, utilizando de uma abordagem qualitativa, para a obtenção dos dados do presente trabalho foi realizada uma pesquisa bibliográfica como procedimento técnico.
        
        
}
\geraintro